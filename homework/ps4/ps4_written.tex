\documentclass[]{article}
\usepackage{lmodern}
\usepackage{amssymb,amsmath}
\usepackage{ifxetex,ifluatex}
\usepackage{fixltx2e} % provides \textsubscript
\ifnum 0\ifxetex 1\fi\ifluatex 1\fi=0 % if pdftex
  \usepackage[T1]{fontenc}
  \usepackage[utf8]{inputenc}
\else % if luatex or xelatex
  \ifxetex
    \usepackage{mathspec}
  \else
    \usepackage{fontspec}
  \fi
  \defaultfontfeatures{Ligatures=TeX,Scale=MatchLowercase}
\fi
% use upquote if available, for straight quotes in verbatim environments
\IfFileExists{upquote.sty}{\usepackage{upquote}}{}
% use microtype if available
\IfFileExists{microtype.sty}{%
\usepackage{microtype}
\UseMicrotypeSet[protrusion]{basicmath} % disable protrusion for tt fonts
}{}
\usepackage[margin=1.5cm]{geometry}
\usepackage{hyperref}
\hypersetup{unicode=true,
            pdftitle={Problem Set 4: Written Part},
            pdfauthor={Your Name Goes Here},
            pdfborder={0 0 0},
            breaklinks=true}
\urlstyle{same}  % don't use monospace font for urls
\usepackage{graphicx,grffile}
\makeatletter
\def\maxwidth{\ifdim\Gin@nat@width>\linewidth\linewidth\else\Gin@nat@width\fi}
\def\maxheight{\ifdim\Gin@nat@height>\textheight\textheight\else\Gin@nat@height\fi}
\makeatother
% Scale images if necessary, so that they will not overflow the page
% margins by default, and it is still possible to overwrite the defaults
% using explicit options in \includegraphics[width, height, ...]{}
\setkeys{Gin}{width=\maxwidth,height=\maxheight,keepaspectratio}
\IfFileExists{parskip.sty}{%
\usepackage{parskip}
}{% else
\setlength{\parindent}{0pt}
\setlength{\parskip}{6pt plus 2pt minus 1pt}
}
\setlength{\emergencystretch}{3em}  % prevent overfull lines
\providecommand{\tightlist}{%
  \setlength{\itemsep}{0pt}\setlength{\parskip}{0pt}}
\setcounter{secnumdepth}{0}
% Redefines (sub)paragraphs to behave more like sections
\ifx\paragraph\undefined\else
\let\oldparagraph\paragraph
\renewcommand{\paragraph}[1]{\oldparagraph{#1}\mbox{}}
\fi
\ifx\subparagraph\undefined\else
\let\oldsubparagraph\subparagraph
\renewcommand{\subparagraph}[1]{\oldsubparagraph{#1}\mbox{}}
\fi

%%% Use protect on footnotes to avoid problems with footnotes in titles
\let\rmarkdownfootnote\footnote%
\def\footnote{\protect\rmarkdownfootnote}

%%% Change title format to be more compact
\usepackage{titling}

% Create subtitle command for use in maketitle
\providecommand{\subtitle}[1]{
  \posttitle{
    \begin{center}\large#1\end{center}
    }
}

\setlength{\droptitle}{-2em}

  \title{Problem Set 4: Written Part}
    \pretitle{\vspace{\droptitle}\centering\huge}
  \posttitle{\par}
    \author{Your Name Goes Here}
    \preauthor{\centering\large\emph}
  \postauthor{\par}
    \date{}
    \predate{}\postdate{}
  
\usepackage{booktabs}

\begin{document}
\maketitle

\def\simiid{\stackrel{{\mbox{\text{\tiny i.i.d.}}}}{\sim}}

\hypertarget{details}{%
\section{Details}\label{details}}

\hypertarget{how-to-write-up}{%
\subsubsection{How to Write Up}\label{how-to-write-up}}

The written part of this assignment can be either typeset using latex or
hand written.

\hypertarget{grading}{%
\subsubsection{Grading}\label{grading}}

5\% of your grade on this assignment is for turning in something
legible. This means it should be organized, and any Rmd files should
knit to pdf without issue.

An additional 15\% of your grade is for completion. A quick pass will be
made to ensure that you've made a reasonable attempt at all problems.

Across both the written part and the R part, in the range of 1 to 3
problems will be graded more carefully for correctness. In grading these
problems, an emphasis will be placed on full explanations of your
thought process. You don't need to write more than a few sentences for
any given problem, but you should write complete sentences!
Understanding and explaining the reasons behind what you are doing is at
least as important as solving the problems correctly.

Solutions to all problems will be provided.

\hypertarget{collaboration}{%
\subsubsection{Collaboration}\label{collaboration}}

You are allowed to work with others on this assignment, but you must
complete and submit your own write up. You should not copy large blocks
of code or written text from another student.

\hypertarget{sources}{%
\subsubsection{Sources}\label{sources}}

You may refer to our text, Wikipedia, and other online sources. All
sources you refer to must be cited.

\hypertarget{problem-i-cheating}{%
\section{Problem I: Cheating}\label{problem-i-cheating}}

This problem is adapted from an exercise in ``Introduction to
Statistical Thought'' by Michael Lavine (2013). Lavine writes:

\begin{quote}
Some researchers are interested in \(\theta\), the proportion of
students who ever cheat on exams. They randomly sample 100 students and
ask ``Have you ever cheated on a college exam?'' Naturally, some
students lie. Let \(\phi_1\) be the proportion of non-cheaters who lie
and \(\phi_2\) be the proportion of cheaters who lie. Let \(X\) be the
number of students who answer ``Yes''.
\end{quote}

Define the following events:

\begin{itemize}
\tightlist
\item
  \(A\) is the event that a randomly sampled student actually has
  cheated on a college exam.
\item
  \(B\) is the event that a randomly sampled student says they have
  cheated on a college exam. (They answer ``Yes''.)
\end{itemize}

\hypertarget{based-on-the-problem-statement-above-write-the-following-probabilities-in-terms-of-the-unknown-parameters-theta-phi_1-and-phi_2.}{%
\subsubsection{\texorpdfstring{(1) Based on the problem statement above,
write the following probabilities in terms of the unknown parameters
\(\theta\), \(\phi_1\), and
\(\phi_2\).}{(1) Based on the problem statement above, write the following probabilities in terms of the unknown parameters \textbackslash{}theta, \textbackslash{}phi\_1, and \textbackslash{}phi\_2.}}\label{based-on-the-problem-statement-above-write-the-following-probabilities-in-terms-of-the-unknown-parameters-theta-phi_1-and-phi_2.}}

\begin{itemize}
\item
  \(P(A)\): the probability that a randomly sampled student has cheated
  on a college exam
\item
  \(P(B | A^c)\): the probability that a randomly sampled student says
  they have cheated on a college exam, given that they actually have not
  cheated
\item
  \(P(B^c | A^c)\): the probability that a randomly sampled student says
  they have not cheated on a college exam, given that they actually have
  not cheated
\item
  \(P(B | A)\): the probability that a randomly sampled student says
  they have cheated on a college exam, given that they actually have
  cheated
\item
  \(P(B^c | A)\): the probability that a randomly sampled student says
  they have not cheated on a college exam, given that they actually have
  cheated
\end{itemize}

\hypertarget{find-pb-the-probability-that-a-randomly-sampled-student-says-they-have-cheated-on-a-college-exam.}{%
\subsubsection{(2) Find P(B), the probability that a randomly sampled
student says they have cheated on a college
exam.}\label{find-pb-the-probability-that-a-randomly-sampled-student-says-they-have-cheated-on-a-college-exam.}}

\hypertarget{specify-a-reasonable-probability-model-for-x-the-number-of-the-100-students-in-the-survey-who-answer-yes.-you-may-assume-that-the-survey-responses-are-independent-for-example-the-respondents-are-not-friends-and-are-not-planning-their-survey-responses-together.-specify-all-parameters-for-your-model-in-terms-of-theta-phi_1-and-phi_2.}{%
\subsubsection{\texorpdfstring{(3) Specify a reasonable probability
model for \(X\), the number of the 100 students in the survey who answer
``Yes''. You may assume that the survey responses are independent; for
example, the respondents are not friends and are not planning their
survey responses together. Specify all parameters for your model in
terms of \(\theta\), \(\phi_1\), and
\(\phi_2\).}{(3) Specify a reasonable probability model for X, the number of the 100 students in the survey who answer ``Yes''. You may assume that the survey responses are independent; for example, the respondents are not friends and are not planning their survey responses together. Specify all parameters for your model in terms of \textbackslash{}theta, \textbackslash{}phi\_1, and \textbackslash{}phi\_2.}}\label{specify-a-reasonable-probability-model-for-x-the-number-of-the-100-students-in-the-survey-who-answer-yes.-you-may-assume-that-the-survey-responses-are-independent-for-example-the-respondents-are-not-friends-and-are-not-planning-their-survey-responses-together.-specify-all-parameters-for-your-model-in-terms-of-theta-phi_1-and-phi_2.}}

\hypertarget{write-down-a-formula-for-the-pdf-of-x-theta-phi_1-phi_2-based-on-your-model-in-part-3.-note-that-you-will-conduct-the-survey-of-100-students-once-and-record-x-x.-you-observe-a-single-x-not-x_1-ldots-x_100.}{%
\subsubsection{\texorpdfstring{(4) Write down a formula for the pdf of
\(X | \Theta, \Phi_1, \Phi_2\) based on your model in part (3). Note
that you will conduct the survey of 100 students once and record
\(X = x\). (You observe a single \(x\), not
\(x_1, \ldots, x_{100}\).)}{(4) Write down a formula for the pdf of X \textbar{} \textbackslash{}Theta, \textbackslash{}Phi\_1, \textbackslash{}Phi\_2 based on your model in part (3). Note that you will conduct the survey of 100 students once and record X = x. (You observe a single x, not x\_1, \textbackslash{}ldots, x\_\{100\}.)}}\label{write-down-a-formula-for-the-pdf-of-x-theta-phi_1-phi_2-based-on-your-model-in-part-3.-note-that-you-will-conduct-the-survey-of-100-students-once-and-record-x-x.-you-observe-a-single-x-not-x_1-ldots-x_100.}}

\hypertarget{suppose-you-have-specified-independent-beta-prior-distributions-for-the-unknown-parameters-theta-phi_1-and-phi_2-theta-sim-textbetaalpha_0-beta_0-phi_1-sim-textbetaalpha_1-beta_1-and-phi_2-sim-textbetaalpha_2-beta_2.-in-this-notation-alpha_0-beta_0-alpha_1-beta_1-alpha_2-and-beta_2-are-numbers-the-analyst-picks-to-specify-the-priors-for-each-of-the-three-parameters-theta-phi_1-and-phi_2.-write-down-formulas-for-the-pdfs-of-these-three-prior-distributions.-you-will-be-writing-down-three-separate-pdfs-involving-theta-phi_1-phi_2-alpha_0-beta_0-alpha_1-beta_1-alpha_2-and-beta_2.}{%
\subsubsection{\texorpdfstring{(5) Suppose you have specified
independent Beta prior distributions for the unknown parameters
\(\theta\), \(\phi_1\), and \(\phi_2\):
\(\Theta \sim \text{Beta}(\alpha_{0}, \beta_{0})\),
\(\Phi_1 \sim \text{Beta}(\alpha_{1}, \beta_{1})\), and
\(\Phi_2 \sim \text{Beta}(\alpha_{2}, \beta_{2})\). In this notation,
\(\alpha_{0}\), \(\beta_{0}\), \(\alpha_{1}\), \(\beta_{1}\),
\(\alpha_{2}\), and \(\beta_{2}\) are numbers the analyst picks to
specify the priors for each of the three parameters \(\theta\),
\(\phi_1\), and \(\phi_2\). Write down formulas for the pdfs of these
three prior distributions. You will be writing down three separate pdfs
involving \(\theta\), \(\phi_1\), \(\phi_2\), \(\alpha_{0}\),
\(\beta_{0}\), \(\alpha_{1}\), \(\beta_{1}\), \(\alpha_{2}\), and
\(\beta_{2}\).}{(5) Suppose you have specified independent Beta prior distributions for the unknown parameters \textbackslash{}theta, \textbackslash{}phi\_1, and \textbackslash{}phi\_2: \textbackslash{}Theta \textbackslash{}sim \textbackslash{}text\{Beta\}(\textbackslash{}alpha\_\{0\}, \textbackslash{}beta\_\{0\}), \textbackslash{}Phi\_1 \textbackslash{}sim \textbackslash{}text\{Beta\}(\textbackslash{}alpha\_\{1\}, \textbackslash{}beta\_\{1\}), and \textbackslash{}Phi\_2 \textbackslash{}sim \textbackslash{}text\{Beta\}(\textbackslash{}alpha\_\{2\}, \textbackslash{}beta\_\{2\}). In this notation, \textbackslash{}alpha\_\{0\}, \textbackslash{}beta\_\{0\}, \textbackslash{}alpha\_\{1\}, \textbackslash{}beta\_\{1\}, \textbackslash{}alpha\_\{2\}, and \textbackslash{}beta\_\{2\} are numbers the analyst picks to specify the priors for each of the three parameters \textbackslash{}theta, \textbackslash{}phi\_1, and \textbackslash{}phi\_2. Write down formulas for the pdfs of these three prior distributions. You will be writing down three separate pdfs involving \textbackslash{}theta, \textbackslash{}phi\_1, \textbackslash{}phi\_2, \textbackslash{}alpha\_\{0\}, \textbackslash{}beta\_\{0\}, \textbackslash{}alpha\_\{1\}, \textbackslash{}beta\_\{1\}, \textbackslash{}alpha\_\{2\}, and \textbackslash{}beta\_\{2\}.}}\label{suppose-you-have-specified-independent-beta-prior-distributions-for-the-unknown-parameters-theta-phi_1-and-phi_2-theta-sim-textbetaalpha_0-beta_0-phi_1-sim-textbetaalpha_1-beta_1-and-phi_2-sim-textbetaalpha_2-beta_2.-in-this-notation-alpha_0-beta_0-alpha_1-beta_1-alpha_2-and-beta_2-are-numbers-the-analyst-picks-to-specify-the-priors-for-each-of-the-three-parameters-theta-phi_1-and-phi_2.-write-down-formulas-for-the-pdfs-of-these-three-prior-distributions.-you-will-be-writing-down-three-separate-pdfs-involving-theta-phi_1-phi_2-alpha_0-beta_0-alpha_1-beta_1-alpha_2-and-beta_2.}}

\hypertarget{suppose-you-now-observe-x-x.-write-down-a-formula-for-the-pdf-of-the-joint-posterior-of-theta-phi_1-and-phi_2-given-x-x-up-to-a-multiplicative-constant-of-proportionality.}{%
\subsubsection{\texorpdfstring{(6) Suppose you now observe \(X = x\).
Write down a formula for the pdf of the joint posterior of \(\Theta\),
\(\Phi_1\), and \(\Phi_2\) given \(X = x\), up to a multiplicative
constant of
proportionality.}{(6) Suppose you now observe X = x. Write down a formula for the pdf of the joint posterior of \textbackslash{}Theta, \textbackslash{}Phi\_1, and \textbackslash{}Phi\_2 given X = x, up to a multiplicative constant of proportionality.}}\label{suppose-you-now-observe-x-x.-write-down-a-formula-for-the-pdf-of-the-joint-posterior-of-theta-phi_1-and-phi_2-given-x-x-up-to-a-multiplicative-constant-of-proportionality.}}

\hypertarget{your-real-goal-is-to-use-the-survey-data-to-learn-about-theta-the-proportion-of-students-who-have-cheated.-phi_1-and-phi_2-are-not-of-direct-interest-they-were-just-necessary-to-get-a-reasonable-model-for-the-data.-parameters-like-this-are-sometimed-referred-to-as-nuisance-parameters.}{%
\subsubsection{\texorpdfstring{(7) Your real goal is to use the survey
data to learn about \(\theta\), the proportion of students who have
cheated. \(\phi_1\) and \(\phi_2\) are not of direct interest, they were
just necessary to get a reasonable model for the data. (Parameters like
this are sometimed referred to as ``nuisance
parameters''.)}{(7) Your real goal is to use the survey data to learn about \textbackslash{}theta, the proportion of students who have cheated. \textbackslash{}phi\_1 and \textbackslash{}phi\_2 are not of direct interest, they were just necessary to get a reasonable model for the data. (Parameters like this are sometimed referred to as ``nuisance parameters''.)}}\label{your-real-goal-is-to-use-the-survey-data-to-learn-about-theta-the-proportion-of-students-who-have-cheated.-phi_1-and-phi_2-are-not-of-direct-interest-they-were-just-necessary-to-get-a-reasonable-model-for-the-data.-parameters-like-this-are-sometimed-referred-to-as-nuisance-parameters.}}

\hypertarget{a.-suppose-you-want-to-calculate-the-posterior-mean-for-the-proportion-of-students-who-have-cheated.-write-this-as-a-suitable-integral-of-the-joint-posterior-pdf-f_theta-phi_1-phi_2-xtheta-phi_1-phi_2-x.}{%
\paragraph{\texorpdfstring{a. Suppose you want to calculate the
posterior mean for the proportion of students who have cheated. Write
this as a suitable integral of the joint posterior pdf
\(f_{\Theta, \Phi_1, \Phi_2 | X}(\theta, \phi_1, \phi_2 | x)\).}{a. Suppose you want to calculate the posterior mean for the proportion of students who have cheated. Write this as a suitable integral of the joint posterior pdf f\_\{\textbackslash{}Theta, \textbackslash{}Phi\_1, \textbackslash{}Phi\_2 \textbar{} X\}(\textbackslash{}theta, \textbackslash{}phi\_1, \textbackslash{}phi\_2 \textbar{} x).}}\label{a.-suppose-you-want-to-calculate-the-posterior-mean-for-the-proportion-of-students-who-have-cheated.-write-this-as-a-suitable-integral-of-the-joint-posterior-pdf-f_theta-phi_1-phi_2-xtheta-phi_1-phi_2-x.}}

Hint: How can you find \(f_{\Theta | X}(\theta | x)\) from
\(f_{\Theta, \Phi_1, \Phi_2 | X}(\theta, \phi_1, \phi_2 | x)\)?

\hypertarget{b.-suppose-you-have-a-sample-theta_1-phi_1_1-phi_2_1-ldots-theta_m-phi_1_m-phi_2_m-from-the-joint-posterior-distribution-of-theta-phi_1-and-phi_2-given-x.-write-how-you-could-approximate-the-integral-from-part-a-using-monte-carlo-integration.}{%
\paragraph{\texorpdfstring{b. Suppose you have a sample
\((\theta_1, {\phi_1}_1, {\phi_2}_1), \ldots, (\theta_m, {\phi_1}_m, {\phi_2}_m)\)
from the joint posterior distribution of \(\Theta\), \(\Phi_1\), and
\(\Phi_2\) given \(X\). Write how you could approximate the integral
from part a using Monte Carlo
integration.}{b. Suppose you have a sample (\textbackslash{}theta\_1, \{\textbackslash{}phi\_1\}\_1, \{\textbackslash{}phi\_2\}\_1), \textbackslash{}ldots, (\textbackslash{}theta\_m, \{\textbackslash{}phi\_1\}\_m, \{\textbackslash{}phi\_2\}\_m) from the joint posterior distribution of \textbackslash{}Theta, \textbackslash{}Phi\_1, and \textbackslash{}Phi\_2 given X. Write how you could approximate the integral from part a using Monte Carlo integration.}}\label{b.-suppose-you-have-a-sample-theta_1-phi_1_1-phi_2_1-ldots-theta_m-phi_1_m-phi_2_m-from-the-joint-posterior-distribution-of-theta-phi_1-and-phi_2-given-x.-write-how-you-could-approximate-the-integral-from-part-a-using-monte-carlo-integration.}}

\hypertarget{c.-suppose-you-want-to-calculate-the-posterior-probability-that-less-than-half-of-students-have-cheated.-write-this-as-a-suitable-integral-of-the-joint-posterior-pdf-f_theta-phi_1-phi_2-xtheta-phi_1-phi_2-x.-set-it-up-so-you-have-an-indicator-function-as-part-of-the-integrand-on-the-inside-of-the-integral.}{%
\paragraph{\texorpdfstring{c. Suppose you want to calculate the
posterior probability that less than half of students have cheated.
Write this as a suitable integral of the joint posterior pdf
\(f_{\Theta, \Phi_1, \Phi_2 | X}(\theta, \phi_1, \phi_2 | x)\). Set it
up so you have an indicator function as part of the integrand (on the
inside of the
integral).}{c. Suppose you want to calculate the posterior probability that less than half of students have cheated. Write this as a suitable integral of the joint posterior pdf f\_\{\textbackslash{}Theta, \textbackslash{}Phi\_1, \textbackslash{}Phi\_2 \textbar{} X\}(\textbackslash{}theta, \textbackslash{}phi\_1, \textbackslash{}phi\_2 \textbar{} x). Set it up so you have an indicator function as part of the integrand (on the inside of the integral).}}\label{c.-suppose-you-want-to-calculate-the-posterior-probability-that-less-than-half-of-students-have-cheated.-write-this-as-a-suitable-integral-of-the-joint-posterior-pdf-f_theta-phi_1-phi_2-xtheta-phi_1-phi_2-x.-set-it-up-so-you-have-an-indicator-function-as-part-of-the-integrand-on-the-inside-of-the-integral.}}

\hypertarget{d.-write-how-you-could-approximate-the-integral-from-part-c-using-monte-carlo-integration.}{%
\paragraph{d. Write how you could approximate the integral from part c
using Monte Carlo
integration.}\label{d.-write-how-you-could-approximate-the-integral-from-part-c-using-monte-carlo-integration.}}


\end{document}
